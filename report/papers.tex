\chapter{Review of papers}

\myminitoc

\sect{Sinkhorn Distances: Lightspeed Computation of Optimal Transport}

\cite{Cut}

\sect{Wasserstein GAN}

\cite{Arj++}

\sect{Convolutional Wasserstein Distances: Efficient Optimal Transportation on Geometric Domains \cite{Goe++}}

Because a single step of Sinkhorn algorithm has a complexity in $\mathcal{O}(n^2)$, computing the Wassertein distance may take a while in many cases. That's why the researchers who write this paper thought about a more effective solution to compute a step. They found a way to use Gaussian convolution with a $\mathcal{O}(n \log n)$ complexity instead of a matrix-vector product. More generally they use a heat kernel. So instead of the convolution it can also be a sparse pre-factored linear system which can be computed in a complexity smaller than $\mathcal{O}(n^2)$. Furthermore by doing this, the convergence is still linear. Of course this can not be applied in all cases of optimal transport, but it can be used on very common geometric domains, like images or meshes. In addition of the computation of Wasserstein distance, they also propose a way to use convolution in the computation of Wasserstein barycentres and Wasserstein propagation.

